\chapter{Introduction}
\section{Machine Learning Category}
\subsection{Parametric versus non-parametric models}
Machine learning algorithms can be grouped into parametric and non-parametric models. Using parametric models, we estimate parameters from the training dataset to learn a function that can classify new data points without requiring the original training dataset anymore. Typical examples of parametric models are the perceptron, logistic regression, and the linear SVM. In contrast, non-parametric models can't be characterized by a fixed set of parameters, and the number of parameters changes with the amount of training data. Two examples of non-parametric models that we have seen so far are the decision tree classifier/random forest and the kernel (but not linear) SVM.

KNN belongs to a subcategory of non-parametric models described as instance-based learning. Models based on instance-based learning are characterized by memorizing the training dataset, and lazy learning is a special case of instance-based learning that is associated with no (zero) cost during the learning process.